\documentclass[../main.tex]{subfiles}

\begin{document}
%% begin abstract format
\makeatletter
\renewenvironment{abstract}{%
    \if@twocolumn
      \section*{Abstract \\}%
    \else %% <- here I've removed \small
    \begin{flushright}
        {\filleft\Huge\bfseries\fontsize{48pt}{12}\selectfont Abstract\vspace{\z@}}%  %% <- here I've added the format
        \end{flushright}
      \quotation
    \fi}
    {\if@twocolumn\else\endquotation\fi}
\makeatother
%% end abstract format
\begin{abstract}

The automatic generation of clinical summaries from unstructured medical texts represents a significant challenge, as redundancy and disorganization in patient records can hinder effective decision-making. This study explores the use of Small Language Models (SLMs) as a cost-effective and accessible alternative to large-scale models, evaluating their performance in clinical summarization tasks. Advanced prompt engineering strategies were applied, with step-by-step generation proving particularly effective. Additionally, a multi-question Retrieval-Augmented Generation (RAG) approach was implemented to enhance contextual understanding. Fine-tuning of the LLaMA 3.2 model revealed effective learning but also stylistic overfitting, highlighting the need for a corpus better aligned with clinical writing. Automatic evaluation was complemented by expert validation, and a functional web application was developed to integrate the optimized model, enabling practical generation and assessment of clinical summaries. Overall, the study validates a reproducible workflow and demonstrates the real-world potential of SLMs in clinical environments, despite their limitations compared to larger models like GPT-4o-mini.

\bfseries{\large{Keywords:}} Clinical summarization, Natural Language Processing (NLP), Small Language Models

\end{abstract}
\end{document}